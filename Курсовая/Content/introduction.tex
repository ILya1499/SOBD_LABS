% Содержание введения
%
Во введении сначала дается краткая характеристика области, в которой выполнена работа (1 -- 3 предложения). Затем обосновывается актуальность работы.

Данная работа выполнена в среде Appache Spark - - это мощный open-source фреймворк для обработки больших данных, который применяется в различных областях:
В целом, область применения Apache Spark очень широка и зависит от потребностей конкретной организации или проекта.

Далее идут фразы, которые лучше повторить дословно:

В связи с этим целью данной работы являлось ... (цель должна быть одна). ?????????????(в двух лабах их несколько)

Для достижения поставленной цели решались следующие задачи:
\begin{enumerate}
\item Познакомиться с понятием «большие данные» и способами их обработки.
\item Познакомиться с инструментом Apache Spark и возможностями, которые он предоставляет для обработки больших данных.
\item Получить представление об инструментах экосистемы Hadoop: HDFS и YARN.
\item Поработать с табличным форматом для больших данных Apache Iceberg.
\item Получить навыки выполнения разведочного анализа данных использованием pyspark.
\end{enumerate}


В конце введения следует добавить описание структуры курсовой работы. Например:
\par В первом разделе рассмотрена более подробно постановка задачи разведовочного анализа датасета с использование фреймворка Apache Spark и библиотеки SparkML
\par Во втором разделе расмотрены более подробно задачи линейной регрессии и бинарной классификации
\par... В третьем разделе ???????????????
\par ... В заключении работы сформулированы общие выводы ???????